\capitulo{Revisão da literatura}
\label{cap:revisao-literatura}

Este capítulo apresenta a revisão da literatura sobre os principais conceitos que fundamentam este trabalho, abordando as tecnologias e os estudos relevantes para a análise de movimento no esporte.

\secao{SOLUÇÕES SIMILARES}
\label{sec:trabalhos-relacionados}

Nesta seção, são apresentados trabalhos que possuem relação com a proposta deste estudo, 
seja pela utilização de técnicas similares ou pela aplicação de \LLM{LLM} no ramo e-commerce ou para busca inteligentes. A seguir, cada um dos trabalhos é abordado individualmente, seguido de uma tabela comparativa ao final.

\subsecao{Fine-tuning BERT for Semantic Textual Similarity with Transformers in Python (Hamdine, 2023)}

O estudo de Hamdine (2023) realizou um fine-tuning no modelo de linguagem BERT para comparação de similaridade textual. Utilizando o benchmark Semantic Textual Similarity \sts{STS} (PAPERSWITHCODE, s.d.) pois possui
diversos conjuntos de dados em inglês voltados para a tarefa de avaliaçãod e semantica.A metodologia aplicada envolve etapas de pré-processamento e preparação dos dados, construção de uma classe personalizada para manipulação do dataset e definição de uma métrica de avaliação adequada.

As estatísticas acima mostram que a perda de similaridade do cosseno de validação
diminuiu continuamente, mostrando que o modelo se tornou cada vez mais preciso na geração de embeddings de frases em correspondência com similaridades semânticas
dos pares de sentenças..

\begin{figura}{Estatísticas obtidas durante o \textit{fine-tuning} do modelo BERT}{HAMDINE, 2023}
    \begin{flushleft}
        \label{fig:Futebol}
        \includegraphics[width=0.48\linewidth]{template-unoesctex/resources/floats/ilustracoes/fine-tuning.png}
    \end{flushleft}
    \addcontentsline{lof}{figure}{Ilustração X – Estatísticas obtidas durante o \textit{fine-tuning} do modelo BERT}
\end{figura}
\FloatBarrier

\subsecao{IMPLEMENTAÇÃO DE BUSCA SEMÂNTICA EM E-COMMERCES COMO APOIO A PESQUISAS INTELIGENTES (Borges, 2024)}

Borges (2024) também realizou o treinamento de um modelo BERT voltado para a tarefa de similaridade textual, 
utilizando para isso um conjunto de dados em português. O modelo foi aplicado em um protótipo de e-commerce de farmácias, 
no qual demonstrou resultados promissores: a partir de buscas simples, foi capaz de recuperar corretamente os 
produtos mais relevantes de acordo com a consulta do usuário.

A Figura 2, apresentada a seguir, ilustra um dos testes realizados após a etapa de fine-tuning conduzida no projeto.

\begin{figura}{ Resultados obtidos da consulta “chocolate” com modelo bertimbau-finetuned, ajustado com
700 amostras e ajuste de hiperparâmetros}{Borges, 2024}
    \begin{flushleft}
        \label{fig:keypoints}
        \includegraphics[width=0.36\linewidth]{resources/floats/ilustracoes/Modelo bert- Hilda.png}
    \end{flushleft}
    \addcontentsline{lof}{figure}{Ilustração X – Resultados obtidos da consulta “chocolate” com modelo bertimbau-finetuned, ajustado com
700 amostras e ajuste de hiperparâmetros}
\end{figura}
\FloatBarrier

\subsecao{Deco AI Sales Assistant: Utilizando LLMs para
experiências de e-commerce personalizadas e
sob demanda (Filho, 2024)}

O estudo de Filho (2024) propôs a integração de um assistente de vendas na plataforma da Deco, empregando três modelos de \llm{LLM} 
desenvolvidos pela OpenAI. A solução buscou facilitar o processo de busca dos usuários, apresentando resultados positivos: 
por meio de interações com o assistente, os usuários foram capazes de encontrar produtos utilizando diferentes 
modalidades de entrada, como áudio, imagem e texto.

A Figura 3 ilustra um dos testes conduzidos, no qual uma persona fictícia denominada Maria, com 68 anos, 
interage com o assistente por meio de comandos de voz, demonstrando a usabilidade do sistema em um cenário prático.

\begin{figura}{Experiência de compra de Maria com Áudio}{Filho, 2024}
    \begin{flushleft}
        \label{fig:esqueleto}
        \includegraphics[width=0.48\linewidth]{resources/floats/ilustracoes/experiencia-maria.png}
    \end{flushleft}
    \addcontentsline{lof}{figure}{Ilustração X – Experiência de compra de Maria com Áudio}
\end{figura}
\FloatBarrier

Ao final dessa seção, a Tabela 1 resume os principais aspectos dos trabalhos e Filho(2024) e de Borges(2024).

Os trabalhos e estudos mencionados discutem a aplicação de modelos de linguagem grandes \llm{LLMs} para aprimorar as experiências de e-commerce, 
focando em busca semântica e assistentes de vendas inteligentes. O trabbalho de Filho(2024) detalha a implementação de 
um assistente de vendas de IA chamado "Deco AI Sales Assistant," que utiliza LLMs da OpenAI para
oferecer recomendações personalizadas e interações conversacionais. O de Borges (2024) mostra a implementação de busca 
semântica em e-commerces farmacêuticos usando técnicas de Processamento de Linguagem Natural \pln{PLN} e o modelo BERTimbau, 
com ênfase no fine-tuning para adaptar o modelo a domínios específicos. Ambos os trabalhos visam otimizar 
a funcionalidade de busca, melhorar a experiência do usuário e, consequentemente, aumentar as taxas de conversão em 
plataformas de comércio.


    \begin{tabela}{Trabalhos Relacionados}{O Autor}
        \label{tab:trabalhos-relacionados}
        \resizebox{\textwidth}{!}{ % Redimensiona a tabela para caber na largura da página
        \begin{tabular}{|c|c|c|c|}
            \hline
            \textbf{TRABALHO}                                                                                                                                           & \textbf{\begin{tabular}[c]{@{}c@{}}OBJETIVO \\ GERAL\end{tabular}}                                                                                                                           & \textbf{TECNOLOGIAS}                                                                                                                                                                                                                                                                                                               & \textbf{\begin{tabular}[c]{@{}c@{}}RESULTADOS \\ ALCANÇADOS\end{tabular}}                                                                         \\ \hline


            \textbf{\begin{tabular}[c]{@{}c@{}}IMPLEMENTAÇÃO \\ DE BUSCA SEMÂNTICA EM\\ E-COMMERCES COMO\\ APOIO A PESQUISAS\\ INTELIGENTES\\ (Borges, 2024)\end{tabular}} & \begin{tabular}[c]{@{}c@{}} Propor uma estratégia de melhoria\\ de resultados de busca\\ em e-commerces. Isso\\ será feito através\\ do desenvolvimento de \\uma ferramenta de busca \\semântica, utilizando \\técnicas de Processamento \\de Linguagem Natural.\end{tabular}                             & \begin{tabular}[c]{@{}c@{}}Utilizou o modelo\\ de linguagem pré-treinado\\ BERTimbau em seu trabalho\end{tabular}                        & \begin{tabular}[c]{@{}c@{}}Um modelo pré treinado\\ para analise textual \\utilizando BERT.\end{tabular}                \\ \hline


            \textbf{\begin{tabular}[c]{@{}c@{}}Deco AI Sales Assistant:\\ Utilizando LLMs\\ para experiências\\ de e-commerce\\ personalizadas e sob demanda\\ (Filho, 2024)\end{tabular}} & \begin{tabular}[c]{@{}c@{}} Utilizar LLMs para experiências\\ de e-commerce personalizadas\\ e sob demanda", é criar \\uma experiência de compras\\ única e personalizada \\para os clientes.\end{tabular}                                        & \begin{tabular}[c]{@{}c@{}}Foi utilizado \\os modelos LLMS\\ da OpenIA:\\ GPT-4 (gpt-4-1106-preview)\\ e GPT-4 Vision (gpt-4-1106-vision-preview).\end{tabular}                        & \begin{tabular}[c]{@{}c@{}}Um assitente eficiente \\que consegue entregar\\ valor para a loja e \\proporciar uam experiencia\\ inovadora ao usuario\end{tabular} \\ \hline


        \end{tabular}}

    \end{tabela}

\vspace{0.5 cm}

Observa-se que o trabalho de Borges (2024) concentrou-se em técnicas de Processamento de Linguagem Natural para busca textual, enquanto Filho (2024) 
explorou assistentes de vendas multimodais baseados em LLMs avançados. Nesse contexto, a presente pesquisa busca um equilíbrio entre essas abordagens, 
ao propor a integração de um modelo de linguagem natural em e-commerce com foco em busca semântica intuitiva, mas sem depender de soluções proprietárias de grande escala.

As pesquisas bibliográficas a seguir foram feitas dentro do Google Acadêmico em Setembro de 2025. Na tabela 4, podemos ver detalhes de resultados obtidos nas pesquisas avançadas de trabalhos relacionados, 
todos os termos foram pesquisados utilizando como pesquisa palavras utilizadas no título de artigos, trabalhos que utilizam a visão computacional tiveram um maior número de resultados, já trabalhos com o tema \ac{yolo} em específico foram mais difíceis de se encontrar.

\begin{tabela}{Resultados das pesquisas por termos no título de artigos}{O Autor}
\label{tab:resultados-pesquisa}
\begin{tabular}{|c|c|c|}
\hline
\textbf{Termos Pesquisados} & \textbf{Período de Tempo} & \textbf{Resultados Obtidos} \\
\hline
Busca inteligentes & 2005 - 2025 & 784.000 resultados \\
Processamento de Linguagem Natural       & 2005 - 2025 & 71.700 resultados  \\
BERTimbau                      & 2005 - 2025 & 1.180 resultados \\
BERTimbau, e-commerce                      & 2005 - 2025 & 118 resultados \\
LLMs, e-commerce        & 2005 - 2025 & 19.800 resultados \\
\hline
\end{tabular}
\end{tabela}
\vspace{0.5cm}



