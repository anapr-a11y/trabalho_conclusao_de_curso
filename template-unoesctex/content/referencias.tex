% Bloco de código para criar o capítulo "Referências" sem numeração
\makeatletter
\ut@section[cap:referencias]{ut@capitulo}{\bfseries\MakeUppercase}{Referências}
\makeatother
\printbibliography

\begin{flushleft}
\setlength{\parskip}{1em}
% 1. Define o espaçamento entre linhas como simples
\renewcommand{\baselinestretch}{1.0}\selectfont
% 2. Define o espaço entre as referências como o de uma linha em branco simples
\setlength{\parskip}{\baselineskip}

AIP PUBLISHING. \textbf{Human Pose Estimation Using BlazePose}. 2023. Disponível em: \url{https://pubs.aip.org/aip/acp/article/2971/1/040049/3296267/Human-pose-estimation-using-blaze-pose}. Acesso em: 21 maio 2025.

AMADIO, A. C.; SERRÃO, J. C. A biomecânica em educação física e esporte. \textbf{Revista Brasileira de Educação Física e Esporte}, v. 25, n. esp., p. 15-24, 2011. Disponível em: \url{https://www.scielo.br/j/rbefe/a/6LRgqXLHGhgyrFMsFG5Vydt/}. Acesso em: 19 maio 2025.

ANALYTICS VIDHYA. \textbf{A Comprehensive Guide on Human Pose Estimation}. 2022. Disponível em: \url{https://www.analyticsvidhya.com/blog/2022/01/a-comprehensive-guide-on-human-pose-estimation/}. Acesso em: 21 maio 2025.

ANTONELLO, R. \textbf{Introdução à Visão Computacional com Python e OpenCV 3}. 2016. Disponível em: \url{http://www.antonello.com.br}. Acesso em: 21 maio 2025.

ARORA, S. et al. \textbf{Theoretical Physics of Deep Learning}. arXiv:2106.10165, 2021. Disponível em: \url{https://arxiv.org/abs/2106.10165}

AWS. \textbf{O que é visão computacional?} 2024. Disponível em: \url{https://aws.amazon.com/pt/what-is/computer-vision/}. Acesso em: 19 maio 2025.

\textbf{BIOMECHANICS of the Basketball Jump Shot}. In: PHYSIO-PEDIA. [S. l.], [s. d.]. Disponível em: \url{https://www.physio-pedia.com/Biomechanics_of_the_Basketball_Jump_Shot}. Acesso em: 11 jun. 2025.

BOESCH, G. \textbf{Human Pose Estimation - Everything You Need to Know}. Viso.ai, 2023. Disponível em: \url{https://viso.ai/deep-learning/pose-estimation-ultimate-overview/}. Acesso em: 21 maio 2025.

BRANDINA, K. \textbf{Biomecânica aplicada ao esporte}. São Paulo: Editora Sol, 2018.

CABRAL, Adriel dos Santos Araújo. \textbf{Sistema de Visão Computacional para Análise de Jogadas em Goalball}. 2024. Trabalho de Conclusão de Curso (Bacharelado em Ciência da Computação) - Universidade Federal da Paraíba, João Pessoa, 2024. Disponível em: \url{https://repositorio.ufpb.br/jspui/handle/123456789/32677}. Acesso em: 02 abr. 2025.

CANAN, F.; MENDES, J. C.; SILVA, R. V. Análise estatística no basquetebol de base: perfil do Campeonato Paranaense de Basquetebol masculino Sub-17. \textbf{Revista Brasileira de Educação Física e Esporte}, v. 29, n. 2, p. 289-302, 2015. Disponível em: \url{https://www.researchgate.net/publication/307763304_Analise_estatistica_no_basquetebol_de_base_perfil_do_Campeonato_Paranaense_de_Basquetebol_masculino_Sub-17}. Acesso em: 17 maio 2025.

CARVALHO, A. S. A. de et al. Análise comparativa entre os principais algoritmos de detecção facial: Haar Cascade, HOG, CNN, YOLO e DeepFace. \textbf{Open Science Research V}, 2022. Disponível em: \url{https://doi.org/10.37885/220709383}. Acesso em: 21 maio 2025.

CASEIRO, A. \textbf{Análise cinemática do lançamento em suspensão do basquetebol}. Dissertação (Mestrado em Ciências do Desporto) - Universidade de Coimbra, 2015. Disponível em: \url{https://estudogeral.uc.pt/handle/10316/96639}. Acesso em: 19 maio 2025.

CASEIRO, A.; COSTA, M. J.; OLIVEIRA, D.; VAZ, J. R.; CASTRO, M. A. Biomechanical parameters of the basketball jump shot: comparison between distances and experience level. \textbf{Acta of Bioengineering and Biomechanics}, v. 25, n. 1, p. 25-33, 2023. DOI: 10.37190/abb-02205-2023-01.

CBB - CONFEDERAÇÃO BRASILEIRA DE BASKETBALL. \textbf{Regras Oficiais de Basketball – FIBA 2020 (versão em português)}. São Paulo: CBB, 2020. Disponível em: \url{https://www.cbb.com.br/wp-content/uploads/Regras-Oficiais-de-Basketball-FIBA-2020-Traduzida-para-Portugues.pdf}. Acesso em: 15 maio 2025.

COB - COMITÊ OLÍMPICO DO BRASIL. \textbf{Basquete}. Disponível em: \url{https://www.cob.org.br/time-brasil/esportes/1-basquete}. Acesso em: 15 maio 2025.

CODERADE.IO. \textbf{Pose Estimation in Computer Vision: Everything You Need to Know}. 2024. Disponível em: \url{https://www.codetrade.io/blog/pose-estimation-in-computer-vision-everything-you-need-to-know/}. Acesso em: 21 maio 2025.

COLETTA, Luiz F. S.; AMARAL, A. C. S. \textbf{Segmentação de Pólipos em Imagens de Colonoscopia utilizando YOLOv8}. 2024. Trabalho de Conclusão de Curso (Graduação em Engenharia de Computação) - Universidade Tecnológica Federal do Paraná, Cornélio Procópio, 2024. Figura 2. Disponível em: \url{https://www.researchgate.net/figure/Figura-2-Arquitetura-simplificada-do-YOLOv8-Jocher-et-al-2023_fig1_381750528}. Acesso em: 12 jun. 2025.

DATA HACKERS. \textbf{Matplotlib e Storytelling com Dados - Parte I}. Medium, 2019. Disponível em: \url{https://medium.com/data-hackers/matplotlib-e-storytelling-com-dados-pt-i-48c289943d60}. Acesso em: 21 maio 2025.

DE ROSE JÚNIOR, D.; GASPAR, M. F.; ASSUMPÇÃO, M. V. Construção e validação preliminar de instrumento de avaliação do desempenho técnico-tático individual no basquetebol. \textbf{Revista de Educação Física/UEM}, v. 16, n. 2, p. 201-208, 2005. Disponível em: \url{https://www.scielo.br/j/refuem/a/FNrcSGnjgGgQ5bHNXMfSqmM/}. Acesso em: 16 maio 2025.

ENCORD. \textbf{YOLO Object Detection Explained: A Beginner's Guide}. 2024. Disponível em: \url{https://encord.com/blog/yolo-object-detection-guide/}. Acesso em: 21 maio 2025.

FONSECA, João José Saraiva da. \textbf{Metodologia da pesquisa científica}. Universidade Estadual do Ceará, 2002.

GERHARDT, Tatiana Engel; SILVEIRA, Denise Tolfo. \textbf{Métodos de pesquisa}. Porto Alegre: Editora da UFRGS, 2009.

GETGURU. \textbf{Computer Vision}. 2024. Disponível em: \url{https://www.getguru.com/pt/reference/computer-vision}. Acesso em: 19 maio 2025.

GIL, Antonio Carlos. \textbf{Métodos e técnicas de pesquisa social}. 6. ed. São Paulo: Atlas, 2008.

GOODFELLOW, I.; BENGIO, Y.; COURVILLE, A. \textbf{Deep Learning}. MIT Press, 2016. Disponível em: \url{https://www.deeplearningbook.org/}

HOMECOURT. \textbf{The future of basketball training is here}. HomeCourt AI, 2025. Disponível em: \url{https://www.homecourt.ai}. Acesso em: 12 jun. 2025.

IMPULSIONA. \textbf{Fundamentos do basquete: conheça os 3 principais}. Disponível em: \url{https://impulsiona.org.br/3-fundamentos-do-basquete/}. Acesso em: 15 maio 2025.

INGRAM MICRO. \textbf{O que é visão computacional e como funciona?} 2024. Disponível em: \url{https://blog.ingrammicro.com.br/inovacao-e-tendencias/visao-computacional}. Acesso em: 19 maio 2025.

JOCHER, Glenn; QIU, Jing. \textbf{Ultralytics YOLOv11}. Versão 11.0.0. [S. l.]: Ultralytics, 2024. Disponível em: \url{https://github.com/ultralytics/ultralytics}. Acesso em: 12 jun. 2025.

KARPATHY, A.; LI, F. F. \textbf{CS231n: Convolutional Neural Networks for Visual Recognition}. Stanford University. Disponível em: \url{https://cs231n.github.io/convolutional-networks/}

Knudson, D. V. (2007). \textbf{Fundamentals of biomechanics}. Springer Science and Business Media. Link: \url{https://link.springer.com/book/10.1007/978-1-4757-2518-2}

LAMAS, L.; MORALES, J. C. P. Integração entre a análise do desempenho e o ensino-aprendizagem nos esportes coletivos. \textbf{Revista Brasileira de Ciências do Esporte}, v. 44, 2022. Disponível em: \url{https://www.scielo.br/j/rbce/a/qxXX8bn463Y8PPMLbQtXFVG/}. Acesso em: 16 maio 2025.

LEES, A. Technique analysis in sports: a critical review. \textbf{Journal of Sports Sciences}, v. 20, p. 813-828, 2002.

LIN, T. et al. \textbf{Towards an Understanding of Situated AR Visualization for Basketball Free-Throw Training}. arXiv preprint arXiv:2104.04118, 2021. Disponível em: \url{https://arxiv.org/abs/2104.04118}. Acesso em: 19 maio 2025.

NASCIMENTO, Jorge Mateus. \textbf{Geração de informações estratégicas para clubes de futebol utilizando visão computacional}. 2023. Trabalho de Conclusão de Curso (Graduação em Análise e Desenvolvimento de Sistemas) - Instituto Federal de Educação, Ciência e Tecnologia Baiano, Teixeira de Freitas, 2023. Disponível em: \url{https://saberaberto.uneb.br/items/f9bd1d98-a62d-4b76-9ba4-bebd26f256b7}. Acesso em: 16 abril 2025.

NOAH BASKETBALL. \textbf{Data-Driven Development}. Noah Basketball, 2025. Disponível em: \url{https://noahbasketball.com}. Acesso em: 12 jun. 2025.

NVIDIA. \textbf{Image Processing vs. Computer Vision: What's the Difference?} 2023. Disponível em: \url{https://resources.nvidia.com/en-us/edge-computing/image-processing-vs-computer-vision}. Acesso em: 21 maio 2025.

OKAZAKI, V. H. A. et al. A review on basketball jump shot. \textbf{Journal of Sports Science and Medicine}, v. 14, n. 3, p. 638-643, 2015. Disponível em: \url{https://www.researchgate.net/publication/279180866_A_review_on_basketball_jump_shot}. Acesso em: 19 maio 2025.

OPENCV. \textbf{OpenCV-Python Tutorials}. [S. l.]: OpenCV, 2025. Disponível em: \url{https://docs.opencv.org/4.x/d6/d00/tutorial_py_root.html}. Acesso em: 12 jun. 2025.

PAIVA, F. A. P. et al. \textbf{Introdução a Python com Aplicações de Sistemas Operacionais}. Natal: Editora IFRN, 2019. ISBN 978-65-86293-38-8.

PAN, W. J. et al. Biomechanical Analysis of Shooting Performance for Basketball Players of Different Genders Based on Computer Vision. \textbf{Journal of Physics: Conference Series}, v. 2024, n. 1, p. 012016, 2021. DOI: 10.1088/1742-6596/2024/1/012016. Disponível em: \url{https://iopscience.iop.org/article/10.1088/1742-6596/2024/1/012016}. Acesso em: 16 abril 2025.

PAPERSWITHCODE. \textbf{Pose Estimation}. 2024. Disponível em: \url{https://paperswithcode.com/task/pose-estimation}. Acesso em: 21 maio 2025.

Pressman, R. S. (2016). \textbf{Engenharia de software: uma abordagem profissional} (7. ed.). AMGH Editora.

REDMON, J. et al. \textbf{You Only Look Once: Unified, Real-Time Object Detection}. arXiv:1506.02640, 2016. Disponível em: \url{https://arxiv.org/abs/1506.02640}. Acesso em: 21 maio 2025.

SOMMERVILLE, Ian. \textbf{Engenharia de Software}. 9ª ed. São Paulo: Pearson Prentice Hall, 2011. Capítulo 8 - Teste de Software.

SOUZA, Werbert de Matos. \textbf{O esporte mediado pelas tecnologias digitais: análise de estudos brasileiros}. 2023. 32 f. Trabalho de Conclusão de Curso (Bacharelado em Educação Física) - Universidade Federal de Pernambuco, Centro Acadêmico de Vitória, Vitória de Santo Antão-PE, 2023.

STRUZIK, A.; PIETRASZEWSKI, B.; ZAWADZKI, J. Biomechanical analysis of the jump shot in basketball. \textbf{Journal of Human Kinetics}, v. 42, p. 73-79, 2014. DOI: 10.2478/hukin-2014-0062.

TEFERI, Getu; ENDALEW, Dessalew. Methods of Biomechanical Performance Analyses in Sport: Systematic Review. \textbf{American Journal of Sports Science and Medicine}, v. 8, n. 2, p. 47-52, 2020. DOI: 10.12691/ajssm-8-2-2.

TEIXEIRA, Jayor. \textbf{Classificação de Golpes de Karatê utilizando Redes Neurais e Visão Computacional}. 2024. Trabalho de Conclusão de Curso (Graduação em Sistemas de Informação) – Universidade Federal de Santa Catarina, Araranguá, 2024. Disponível em: \url{https://repositorio.ufsc.br/handle/123456789/262114}. Acesso em: 16 abril 2025.

ULTRALYTICS. \textbf{A aplicação da visão computacional no esporte}. 2024. Disponível em: \url{https://www.ultralytics.com/pt/blog/exploring-the-applications-of-computer-vision-in-sports}. Acesso em: 19 maio 2025.

ULTRALYTICS. Models. In: ULTRALYTICS. \textbf{Ultralytics YOLOv8 Documentation}. [S. l.], 2025. Disponível em: \url{https://docs.ultralytics.com/pt/models/}. Acesso em: 12 jun. 2025.

UNIVERSIDADE FEDERAL DE SANTA MARIA. \textbf{Métodos de Medição: Cinemetria}. Santa Maria: UFSM, 2019. Disponível em: \url{https://www.ufsm.br/app/uploads/sites/644/2019/07/metodos_de_medicao_cinemetria.pdf}. Acesso em: 12 abr. 2025.

YASEEN, Muhammad. \textbf{What is YOLOv8: An In-Depth Exploration of the Internal Features of the Next-Generation Object Detector}. arXiv preprint arXiv:2408.15857, 2024. Disponível em: \url{https://arxiv.org/abs/2408.15857}. Acesso em: 14 abr. 2025.

\end{flushleft}