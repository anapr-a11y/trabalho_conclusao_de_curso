\capitulo{Introdução}
\label{cap:introducao}

A utilização da inteligência artificial tem se expandido de maneira significativa em diferentes setores do mercado. As organizações vêm incorporando essas ferramentas tanto em processos internos, 
visando maior eficiência operacional, quanto em soluções voltadas ao usuário final, proporcionando novas experiências de interação.No contexto de e-commerce, por exemplo, os mecanismos de busca tradicionais, 
baseados em filtros rígidos, podem se mostrar limitados e pouco intuitivos.

Nesse cenário, modelos de linguagem de grande escala (LLMs) têm se destacado ao possibilitar a conversão de linguagem natural em consultas estruturadas (text-to-SQL), 
o que favorece buscas mais personalizadas e acessíveis. Essa evolução contribui para a simplificação da experiência do usuário, tornando o processo de encontrar produtos e serviços mais eficiente. 

A empresa Amo Delivery, fundada em 2017 e sediada em Chapecó, consolidou-se como uma plataforma dedicada a melhorar a experiência dos usuários no setor alimentício. 
Atualmente presente em mais de 300 cidades do Brasil, sua solução de software, denominada “Amo Ofertas”, possibilita ao usuário acessar, de maneira prática e centralizada, 
ofertas de diferentes restaurantes, trazendo conveniência e agilidade ao processo de escolha.

Diante desse contexto, o presente trabalho propõe o estudo e na implementação de um sistema de busca inteligente no aplicativo Amo Ofertas, com base em modelos de linguagem natural. 
O sistema permitirá que os usuários realizem pesquisas em linguagem natural, eliminando a necessidade de filtros rígidos e comandos específicos, ao converter as consultas em 
linguagem SQL estruturada (text-to-SQL). Dessa forma, o usuário poderá buscar por ofertas de forma intuitiva, utilizando expressões semelhantes à conversação cotidiana.

Além disso, a solução contemplará a integração de recursos de reconhecimento de voz (audio-to-text), possibilitando que as consultas sejam feitas por comandos de voz. Essa funcionalidade
tem como objetivo ampliar a acessibilidade e proporcionar maior comodidade, permitindo que os usuários interajam com a aplicação em diferentes contextos de uso, como em 
situações em que a digitação não seja prática.

A implementação dessas funcionalidades no contexto de busca e consulta de ofertas dentro da plataforma Amo Ofertas, 
não irão contemplar modificações nos processos de pagamento, logística de entrega ou outros módulos já existentes. O foco deste trabalho 
permanecerá no aprimoramento da experiência de busca do usuário, por meio da aplicação de técnicas de 
inteligência artificial voltadas à interação natural e personalizada.


\secao{Problema de pesquisa e justificativa}
\label{sec:problema-pesquisa-justificativa}

A análise do desempenho de atletas no basquetebol por meio de \ac{cv} é um campo promissor, 
pois permite uma avaliação dos movimentos realizados durante o jogo. No entanto, a implementação dessa tecnologia enfrenta desafios, 
como a captação eficiente dos dados visuais, o processamento das informações e a interpretação precisa dos resultados para oferecer um feedback útil aos atletas e treinadores. 
Conforme Cabral (2024), a \ac{cv} tem sido amplamente empregada no esporte para melhorar o desempenho dos atletas e apoiar decisões estratégicas, impactando diretamente na performance em quadra.

O problema central desta pesquisa reside na necessidade de um protótipo de \ac{cv} para a análise biomecânica do arremesso no basquetebol, 
especialmente voltado para um time amador que compete regionalmente. A maior parte dos estudos atuais foca em atletas de alto rendimento, 
deixando lacunas para o desenvolvimento de tecnologias aplicáveis a contextos não profissionais. 
Segundo Souza (2023), há uma carência de estudos que abordem a inserção tecnológica no esporte de maneira acessível, 
especialmente no Brasil, onde ainda existem barreiras como desigualdade e falta de recursos.

A implementação de uma ferramenta baseada em \ac{cv} poderá auxiliar treinadores e jogadores a compreender melhor os erros e acertos no movimento de arremesso, 
auxiliando na correção da angulação corporal e do braço do atleta, e também na velocidade do arremesso, contribuindo para um maior aproveitamento de cestas, consequentemente, 
aprimorando também o desempenho esportivo. Além disso, Cabral (2024) destaca que modelos de deep learning podem ser empregados para o reconhecimento automático de padrões de movimento, 
permitindo uma análise detalhada e precisa dos gestos esportivos.



\secao{Objetivos}
\label{sec:objetivos}
Nesta seção serão abordados os objetivos gerais e específicos a serem buscados no deccorer da execução do trabalho proposto.

\subsecao{Objetivo geral}
\label{ssec:objetivo-geral}

Desenvolver um protótipo baseado em visão computacional para analisar biomecanicamente o arremesso no basquetebol, com o objetivo de fornecer correções técnicas do gesto esportivo.

\subsecao{Objetivos específicos}
\label{ssec:objetivos-especificos}

Os objetivos específicos do projeto seguem a ordem cronológica de execução das atividades, conforme descrito a seguir:

\begin{myitemize}
    \item Capturar vídeos de arremessos realizados por atletas da linha de lance livre, com ângulos adequados para a análise biomecânica.
    
    \item Importar os vídeos capturados para o sistema e aplicar técnicas de visão computacional para detectar os pontos corporais do atleta, com ênfase no braço de arremesso, utilizando o modelo yolo.

    \item Calcular os ângulos articulares e parâmetros cinemáticos com base nos pontos detectados.

    \item Comparar os dados extraídos com parâmetros biomecânicos de referência previamente definidos.

    \item Integrar os resultados a uma interface simples e intuitiva, permitindo a visualização e interpretação dos dados obtidos.

    \item Identificar diferenças técnicas entre os arremessos analisados e o modelo ideal, sugerindo correções técnicas com base no feedback automático do sistema.
\end{myitemize}