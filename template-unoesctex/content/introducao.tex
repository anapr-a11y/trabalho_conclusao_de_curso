\capitulo{Introdução}
\label{cap:introducao}

A utilização da inteligência artificial \ia{IA} vem revolucionado de maneira significativa em diferentes setores do mercado. Desde novembro de 2022, com o lançamento
do ChatGPT a população teve acesso ao que era exclusivo somenteas pessoas que trabalhavam na área, depois desse episodio, a maneira de pesquisar/buscar mudou completamente de 
patamar, hoje estudos mostram que o novo formato de interagir com os dados está cada vez mais utilizados e aceito pela geraçãos. 

As organizações vêm incorporando essas ferramentas tanto em processos internos, 
visando maior eficiência operacional, quanto em soluções voltadas ao usuário final, proporcionando novas experiências de interação.
No contexto de e-commerce, por exemplo, os mecanismos de busca tradicionais, baseados em filtros rígidos, podem se mostrar limitados e pouco intuitivos.

Nesse cenário, modelos de linguagem de grande escala \llm{LLM} têm se destacado ao possibilitar a conversão de linguagem natural em consultas estruturadas (text-to-SQL), 
o que favorece buscas mais personalizadas e acessíveis. Essa evolução contribui para a simplificação da experiência do usuário, tornando o processo de encontrar produtos e serviços mais eficiente. 

A empresa Amo Delivery, fundada em 2017 e sediada em Chapecó, consolidou-se como uma plataforma dedicada a melhorar a experiência dos usuários no setor alimentício. 
Atualmente presente em mais de 300 cidades do Brasil, sua solução de software, denominada “Amo Ofertas”, possibilita ao usuário acessar, de maneira prática e centralizada, 
ofertas de diferentes restaurantes, trazendo conveniência e agilidade ao processo de escolha.

Diante desse contexto, o presente trabalho propõe o estudo e na implementação de um sistema de busca inteligente no aplicativo Amo Ofertas, com base em modelos de linguagem natural. 
O sistema permitirá que os usuários realizem pesquisas em linguagem natural, eliminando a necessidade de filtros rígidos e comandos específicos, ao converter as consultas em 
linguagem SQL estruturada (text-to-SQL). Dessa forma, o usuário poderá buscar por ofertas de forma intuitiva, utilizando expressões semelhantes à conversação cotidiana.

A implementação dessas funcionalidades no contexto de busca e consulta de ofertas dentro da plataforma Amo Ofertas, 
não irão contemplar modificações nos processos de pagamento, logística de entrega ou outros módulos já existentes. O foco deste trabalho 
permanecerá no aprimoramento da experiência de busca do usuário, por meio da aplicação de técnicas de 
inteligência artificial voltadas à interação natural e personalizada.


\secao{Problema e justificativa}
\label{sec:problema-pesquisa-justificativa}

A experiência do usuário em plataformas de e-commerce está diretamente relacionada à sua capacidade de 
encontrar, de forma rápida e intuitiva, os produtos que deseja. Com o avanço da tecnologia, o uso de
 modelos de linguagem natural \lmm{LLMS} tornou-se uma tendência crescente, oferecendo interações mais 
 inteligentes e próximas da linguagem cotidiana dos consumidores. Estudos indicam que a eficácia 
 dos mecanismos de busca em plataformas digitais impacta de maneira significativa tanto a satisfação 
 do usuário quanto a concretização das vendas (JANSEN; MOLINA, 2006).

Empresas consolidadas, como Google, Nubank e C6 Bank, já incorporam tecnologias de processamento de 
linguagem natural \pln{PLN} em seus serviços, elevando o nível da experiência do usuário. 
No setor de e-commerce, organizações como Amazon, Shopify e iFood vêm explorando recursos de 
aprendizado de máquina (do inglês ,Machine Learning \ml{ML}) para tornar a busca mais personalizada e acessível, permitindo que 
os usuários encontrem produtos de forma simples e eficiente.

Apesar do crescimento do comércio eletrônico e da evolução dos sistemas de busca baseados em 
\ia{IA}, muitos aplicativos ainda oferecem experiências limitadas ao usuário, especialmente em relação à forma de localizar produtos. Os filtros
de pesquisa são essencias para permitir que o usuario localize oque deseja com base me criterios especificos para sua solicitação
como, o preço, igredienete , distancia, entre outros atributos. 

O aplicativo AMO Ofertas é utiliza filtros rígidos e pouco flexíveis, 
o que compromete a experiência do usuário e pode limitar sua competitividade no mercado. 
Diante disso, este trabalho justifica-se pela necessidade de implementar um modelo de 
linguagem natural que permita buscas mais inteligentes e personalizadas, ampliando a 
satisfação do consumidor, fortalecendo a posição da empresa no setor e possibilitando uma 
concorrência mais equilibrada com players de maior porte.


\secao{Objetivos}
\label{sec:objetivos}
Nesta seção serão abordados os objetivos gerais e específicos a serem buscados no deccorer da execução do trabalho proposto.

\subsecao{Objetivo geral}
\label{ssec:objetivo-geral}

Implementar e avaliar um modelo de linguagem natural /lmm{LLM} no aplicativo AMO Ofertas, com o propósito de 
aprimorar o mecanismo de busca, tornando-o mais intuitivo e eficiente, de modo a melhorar 
a experiência do usuário e aumentar a competitividade da plataforma no mercado de e-commerce.

\subsecao{Objetivos específicos}
\label{ssec:objetivos-especificos}

Os objetivos específicos do projeto seguem a ordem cronológica de execução das atividades, conforme descrito a seguir:

\begin{myitemize}
    \item Estudar a estrutura do sistema de busca existente e documentar as limitações dos filtros rígidos atualmente oferecidos no sistema Amo Ofertas.
    
    \item Coletar logs/chaves de pesquisas realizadas no aplicativo ou aplicar questionários/testes com usuários para identificar 
    padrões de busca, como termos mais usados, erros frequentes afim de analisar dados de busca dos usuários.
      
    \item Realizar um estudo comparativo de desempenho dos modelos de linguagem natural, requisitos computacionais e aderência ao português, 
     justificando a escolha do modelo mais adequado para o contexto da Amo Ofertas.
    
    \item Implementar uma versão experimental em ambiente controlado (Sandbox) para realizar testes isolada para treinar e validar o modelo escolhido,
      simulando consultas reais de usuários.
    
    \item Treinar e ajustar o modelo para o domínio do e-commerce da Amo Ofertas afim de melhorar a relevância e precisão das respostas de busca.
    
    \item Implementar o protótipo dentro do aplicativo, disponibilizar a busca inteligente para um grupo de usuários-teste e coletar feedback qualitativo e quantitativo.
    
    \item Medir métricas como precisão das buscas, tempo médio para encontrar produtos, satisfação do usuário e comparar com o sistema de busca anterior.
\end{myitemize}